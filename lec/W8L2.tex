\documentclass{article}
\usepackage{graphicx}
\usepackage{xcolor}
\usepackage{amsmath}
\usepackage{amsfonts}
\usepackage{amssymb}
\usepackage{amsthm}
\usepackage{tikz}
\usepackage{pgfplots}
\usepackage{listings}
\usepackage{algpseudocode}
\usepackage{xparse}
\usepackage{ebproof}
\usepackage{mathptmx}
% \usepackage{fontspec}
\usepackage[english]{babel}

\def\pl{\partial}
\def\dis{\displaystyle}
\def\rarr{\rightarrow}

% \newcommand{\newmacro}[nun_args]{<expr> #1 <expr> #2 ...}
\DeclareMathAlphabet{\altmathcal}{OMS}{cmsy}{m}{n}

\newcommand{\uset}[2]{\underset{#1}{#2}}
\newcommand{\mb}[1]{\mathbf{#1}}
\newcommand{\tb}[1]{\textbf{#1}}
\newcommand{\mm}[1]{\mathbb{#1}}
\newcommand{\mr}[1]{\mathrm{#1}}
\newcommand{\abs}[1]{|| #1 ||}

\newcommand{\cdblk}[2]{
	\begin{lstlistings}[language=#1]
		#2
	\end{lstlistings}
}

\newcommand{\cword}[1]{\texttt{\textcolor{blue}{#1}}}

\theoremstyle{definition}
\newtheorem{definition}{Definition}[section]

\theoremstyle{remark}
\newtheorem*{remark}{Remark}

\title{W8L2}
\author{ntatsu}
\date{April 10, 2024}

\begin{document}

\maketitle

\section{NP-completeness}

% ======================================================
% begin here
% ======================================================
Recall that proving NP-completeness is the conjunction of proving some problem in NP, and is also NP-hard

\begin{itemize}
    \item Show that problem has P certificate
    \item Show that there exists P transformation to NP-hard problem
\end{itemize}

\subsection{Moar Reductions}

\subsubsection*{CSAT}
\begin{definition}[CSAT]
    CSAT is SAT except its clauses are in conjunctive normal form (CNF, not to be confused with Chomsky Normal Form)
\end{definition}

So far this is still not very obvious, we could consider another special case of CSAT being 3SAT which we should have memory of. \\
Reduction from SAT to 3SAT is talked about in \tb{3821}. We ommit it here but we can do a quick vocal recall.
\begin{itemize}
    \item any clauses of length 1 to 3 have trivial transformation
    \item any clauses of legnth greater than 3 will take the form $\{t_1, t_2, s_1\}\{\neg s_1, t_4, s_2\}\{\neg s_2, t_6, s_3\}\dots$
    \item where $s_1, s_2, \dots$ is dependent on the truth value of the previously omitted clauses
\end{itemize}

\subsubsection*{nature of NP}
An NP problem always has a corresponding non-deterministic TM that matches its solution space.

\subsubsection*{Vertex Cover}
Reduce 3SAT to VC
\begin{itemize}
    \item $n$ vows
    \item $m$ clauses
    \item Truth assignment
    \begin{itemize}
        \item assign truth values, if the vertex is covered or not
        \item we can recall to \tb{tut8} reduction of 3COL to 3SAT
    \end{itemize}
\end{itemize}

Steps we can develop
\begin{itemize}
    \item $n$ vows and $m$ clauses
    \item full proof on \tb{pg.9}
    \item variable gadgets should have an edge connecting two variables taking the opposite negation
    \item clause gadgets should take $3m$ vertices and $3m$ edges.
    \item the resulting graph would have $2n + 3m$ vertices.
    \item an example of the clause $(x \vee \neg y \vee z)$ would take the shape of
    \begin{itemize}
        \item a central triangle clause
        \item the nodes of the clauses are connecting to the corresponding variable gadgets to its corresponding side (i.e. $x$ to $x$, $\neg y$ to $\neg y$ etc.)
    \end{itemize}
    \item If the truth assignment is satisfiable, then the vertex cover of $k$ can achieved
    \item conversely, if the vertex cover should cover at least two of the three of a triangular clause, it is guaranteed that the two covered vertices are not of the same variables with opposing negations.
\end{itemize}

\subsubsection*{Clique}
Knowing the relation between independent set and vertex cover, we can ensure that
\begin{itemize}
    \item $\uset{(G, K)}{\text{VC}} \leq_p \uset{\bar{G}, n - k}{\text{CLIQUE}}$
\end{itemize}

\subsection{Hamiltonian Path}
Reductions from Hamiltonian to 3SAT is unintuitive, so we can introduce an intermediate problem,
\begin{definition}[Zig-zag]
    The $n$ variable gadgets will total $(4 + 2m)n \ \text{gadgets}, (4m + 8)n \ \text{edges}$. The $m$ clause vertices will serve as the "MITM" where a zig or a zag will shortcut and go through a clause correspondingly, and $m$ is directed. For example, if that clause is setup for a zig, then if a zag shortcuts through the clause, it will return to a position that is a subset of that zag, indicating that it has been covered already, negating the truth value of this Hamiltonian path (since it is no longer one).
\end{definition}

\section{Complexity Class}

\begin{itemize}
    \item coNP is like NP, except the P accepter is replaced by a P rejecter.
\end{itemize}

\end{document}

