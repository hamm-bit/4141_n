\documentclass{article}
\usepackage{graphicx}
\usepackage{xcolor}
\usepackage{amsmath}
\usepackage{amsfonts}
\usepackage{amssymb}
\usepackage{amsthm}
\usepackage{tikz}
\usepackage{pgfplots}
\usepackage{listings}
\usepackage{algpseudocode}
\usepackage{xparse}
\usepackage{ebproof}
\usepackage{mathptmx}
% \usepackage{fontspec}
\usepackage[english]{babel}

\def\pl{\partial}
\def\dis{\displaystyle}
\def\rarr{\rightarrow}

% \newcommand{\newmacro}[nun_args]{<expr> #1 <expr> #2 ...}
\DeclareMathAlphabet{\altmathcal}{OMS}{cmsy}{m}{n}

\newcommand{\uset}[2]{\underset{#1}{#2}}
\newcommand{\mb}[1]{\mathbf{#1}}
\newcommand{\tb}[1]{\textbf{#1}}
\newcommand{\mm}[1]{\mathbb{#1}}
\newcommand{\mr}[1]{\mathrm{#1}}
\newcommand{\abs}[1]{|| #1 ||}

\newcommand{\cdblk}[2]{
	\begin{lstlistings}[language=#1]
		#2
	\end{lstlistings}
}

\newcommand{\cword}[1]{\texttt{\textcolor{blue}{#1}}}

\theoremstyle{definition}
\newtheorem{definition}{Definition}[section]

\theoremstyle{remark}
\newtheorem*{remark}{Remark}

\title{Ruleset}
\author{ntatsu}
\date{April 09, 2024}

\begin{document}

\maketitle

\section{Grammar conversion to CNF}
Checking if the following words are in $G$ involves the use of the CYK algorithm. However $R$ is not in CNF, so we have to convert them first as follows.

Steps involved:

\begin{itemize}
    \item Remove $\epsilon$
    \begin{itemize}
        \item There's no epsilon, we simply add singular terms into $E$
    \end{itemize}
    \item Remove all non-singular terminal symbols
    \begin{itemize}
        \item $E_0 \rightarrow E$
        \item $E \rightarrow PEE | MEE | NE | K | 0 | 1 | 2 | I | x | y$
        \item $K \rightarrow 0 | 1 | 2$
        \item $I \rightarrow x | y$
        \item $P \rightarrow +$
        \item $M \rightarrow *$
        \item $N \rightarrow -$
    \end{itemize}
    \item Remove singleton rules where RHS is non-terminal
    \begin{itemize}
        \item $E_0 \rightarrow PEE | MEE | NE | 0 | 1 | 2 | x | y$
        \item $E \rightarrow PEE | MEE | NE | 0 | 1 | 2 | x | y$
        \item $P \rightarrow +$
        \item $M \rightarrow *$
        \item $N \rightarrow -$
        \item Since $K$ and $I$ is not paired with any other non-terminal symbols, they can be removed altogether from the $R$.
    \end{itemize}
    \item Reduce RHS to pairs of non-terminals
    \begin{itemize}
        \item $E_1 \rightarrow EE$
        \item $E_0 \rightarrow PE_1 | ME_1 | NE | 0 | 1 | 2 | x | y$
        \item $E \rightarrow PE_1 | ME_1 | NE | 0 | 1 | 2 | x | y$
        \item $P \rightarrow +$
        \item $M \rightarrow *$
        \item $N \rightarrow -$
    \end{itemize}
    \item Hence the above ruleset is the CNF of the given grammar.
\end{itemize}

\section{Derivation}

The given grammar illustrates the Polish notation for arithmetics. We could be higher-level about this, instead of using the CYK algorithm on a strict CFG sense. We have,

% ======================================================
% begin here
% ======================================================
\subsection{1)}
The following can be derived by syntax tree deduction. Deduction constructs a viable deduction of the given word from the syntax.
\[
\begin{prooftree}
                \hypo{}
            \infer 1 {1 \enskip \quad E}
                \hypo{}
            \infer 1 {1 \enskip \quad E}
        \infer 2 {+ 1 y \enskip \quad PEE}
                    \hypo {}
                \infer 1 {2 \enskip \quad E}
            \infer 1 {-2 \enskip \quad NE}
                \hypo {}
            \infer 1 {0 \enskip \quad E}
        \infer 2 {* -2 0 \enskip \quad MEE}
    \infer 2 {+ + 1 * -2 0 \enskip \quad PEE}
\end{prooftree}
\]

\subsection{2)}
\[
\begin{prooftree}
            \hypo {}
        \infer 1 {1 \enskip \quad E}
            \hypo {* x \enskip \quad N/A}
                \hypo {}
            \infer 1 {x \enskip \quad E}
        \infer 2 {* * x x \enskip \quad MEE}
    \infer 2 {+ 1 * * x x \enskip \quad PEE}
\end{prooftree}
\]
One branch of the above syntax tree cannot reach an endpoint. Hence the word is not part of this grammar.

\subsection{3)}
\[
\begin{prooftree}
            \hypo{}
        \infer 1 {0 \enskip \quad E}
                \hypo {}
            \infer 1 {2 \enskip \quad E}
            \hypo {1 0 \enskip \quad N/A}
        \infer 2 {-2 1 0 \enskip \quad NE}
    \infer 2 {* 0 - *2 1 0 \enskip \quad MEE}
\end{prooftree}
\]
One branch of the above syntax tree cannot reach an endpoint. Hence the word is not part of this grammar.

\end{document}

