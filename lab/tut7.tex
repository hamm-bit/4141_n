\documentclass{article}
\usepackage{graphicx}
\usepackage{xcolor}
\usepackage{amsmath}
\usepackage{amsfonts}
\usepackage{amssymb}
\usepackage{amsthm}
\usepackage{tikz}
\usepackage{pgfplots}
\usepackage{listings}
\usepackage{algpseudocode}
\usepackage{xparse}
\usepackage[english]{babel}

\def\pl{\partial}
\def\dis{\displaystyle}
\def\rarr{\rightarrow}

% \newcommand{\newmacro}[nun_args]{<expr> #1 <expr> #2 ...}

\newcommand{\uset}[2]{\underset{#1}{#2}}
\newcommand{\mb}[1]{\mathbf{#1}}
\newcommand{\tb}[1]{\textbf{#1}}
\newcommand{\mm}[1]{\mathbb{#1}}
\newcommand{\mr}[1]{\mathrm{#1}}
\newcommand{\abs}[1]{|| #1 ||}

\newcommand{\cdblk}[2]{
	\begin{lstlistings}[language=#1]
		#2
	\end{lstlistings}
}

\newcommand{\cword}[1]{\texttt{\textcolor{blue}{#1}}}

\theoremstyle{definition}
\newtheorem{definition}{Definition}[section]

% \theoremstyle{proof}

\theoremstyle{remark}
\newtheorem*{remark}{Remark}

\title{tut7}
\author{ntatsu}
\date{March 27, 2024}

\begin{document}

\maketitle

\section{Turing}

% ======================================================
% begin here
% ======================================================

CFG equivalence can be solved by converting it into turing machines

\subsection{PL for CFG}
\begin{itemize}
    \item $|vy| \geq 1$,
    \item $|vxy| \leq p$,
    \item $uv^nxy^nz \in L \forall n \geq 0$.
\end{itemize}

\subsection{CNF}
For $A, B, C$ cannot be terminal strings,
\begin{itemize}
    \item $A \rightarrow BC$
    \item $A \rightarrow a$
    \item $S \rightarrow \epsilon$
\end{itemize}

\subsection{CFL example from lecture}
\subsubsection*{Example}
Let $L_1 = \{a^ib^ic^i | i > 0\}$. Prove that this is not a CFL. Let $p$ be pumping length, $w = a^pb^pc^p$
\begin{itemize}
    \item if $v^nxy^n = a^*b^*$, then $uxz \notin L$, given that $|vy| \geq 1$, then $uxz$ would have less than $a$ and $b$ and $p$ c, not an element of $L$.
    \item similarly if $v^nxy^n = b^*c^*$
    \item Hence not a CFL
\end{itemize}

\subsubsection*{Example}
Let $L_1 = \{ww | w \in {0,1}^*\}$. Prove that this is not a CFL. Let $p$ be pumping length, take $w = 0^p1^p0^p1^p \in L$
\begin{itemize}
    \item if $v^nxy^n = 1^*0^*$, then given that $|vy| \geq 1$. Then $uxz$ would have $0^p1^{p-n}0^{p-m}1^p$, for $n, m : n + m \leq p$, where $=$ is achieved when $x = \epsilon$.
    \item similarly if $v^nxy^n = 0^*1^*$, $uxz$ is not in $L$ regardless of what $v, x, y$ takes.
    \item Hence not a CFL
\end{itemize}

\subsection{1b}
\begin{definition}[Turing machines]
    Define $\mr{TM : M}$ on input $\langle G_1, G_2 \rangle$
\end{definition}

\begin{itemize}
    \item Use $CYK$ to check if $w \in L(G_1)$ and/or $L(G_2)$
    \item If $w$ is in one language but not both, \tb{ACCEPT}
    \item Else- kepping looking through $w$
\end{itemize}

\subsection{1c}
$U_{CFG}$ is undecidable as $U_{CFG}$ is undecidable. \\ 
We show $L$ is undecidable by showing $\bar{U_{CFG}}$
\begin{proof}[Proof of claim]
    On input $\langle G \rangle$ for $\bar{U_{CFG}}$, construct $\langle G_1, G_2 \rangle$ as
    \begin{align*}
        G_1 &= G_r \\
        G_2 &= G_r \text{where} G_r: S \rarr \epsilon|ss|0|1
    \end{align*}
    Since $L(G_r)$ = \Sigma^*
    \begin{align*}
        \langle G \in \bar{U_{CFG}} \infers \langle G_1, G_r \rangle \in L
        \langle G \in \bar{U_{CFG}} \infers \langle G_2, G_r \rangle \notin L
    \end{align*}
    $\therefore L$ is undecidable.
\end{proof}

\subsection{3a}
No - you can write a TM for a decidable language which does not decide the language (by infinite loops instead of \tb{REJECT})

\subsection{3b}
\[
    A_{TM} : \{\langle M, w \rangle : M is a TM that accepts w \}
\]
On input $\langle M, w \rangle$, construct $M_1$,
\begin{itemize}
    \item Definition of $M_1$
    \item On input $x$
    \begin{itemize}
        \item Ignore $x$
        \item Run $M$ on input $w$
        \item If $M$ accepts, \tb{ACCEPT}
        \item If $M$ rejects, \tb{infinite loop}
    \end{itemize}
    \item If $w \in L(M)$, $M_1$ is a decider (if \tb{ACCEPTING})
    \item If $w \notin L(M)$, $M_1$ is not a decider (either simulating infinite loops or $\mb{M_1}$ infinite loops)
\end{itemize} 


\end{document}

